
\textbf{TLS certificate}

I. Torroledo等人(2018)实现了第一个利用TLS协议的钓鱼攻击检测。由于TLS协议中数据报被加密,因此难以利用数据信息进行检测。该方法利用TLS协议中的证书信息进行钓鱼攻击检测,作者依据其所在单位多年的经验总结了35个与TLS证书相关的特征,使用单热编码方案对特征进行编码,编码后的向量输入神经网络进行学习。

神经网络由三部分并行构成,两部分是LSTM网络,另一部分是简单的ReLu激活函数。

该方法实现了94.87\%的恶意软件精确率,88.64\%的钓鱼网站精确率。该方法的不足之处是对于钓鱼网站的检测率较低,这是由于钓鱼网站的证书会与伪装得真实的证书十分相近。

\textbf{Cyber-Physical System Networks}

Peter Schneider等人(2018)提出了一个可对多种现场总线协议进行异常检测的实时、统一平台。该方法直接利用线路中的字节流数据进行异常检测,因此对多种上层协议有通用性。作者提出的检测方法是,将线路中的数据包按照定长进行划分,将每一个划分的字节流作为向量输入到层叠的去噪自动编码机中。其中,层叠指隐含层是多层的,去噪是指在训练时会对训练记得数据人工添加噪声,以避免过拟合。

自动编码机的架构包含编码机和解码机,训练时二者成对的被训练。损失函数定义为由解码机恢复的向量与原向量之间均方误差。对于正常的字节流,该编码机输出的向量与原始数据之间会有很小的均方误差;而对于异常数据,编码机的输出会有明显的误差。在给定阈值的情况下,只要是高于阈值的数据流都被判断为存在异常。

由于该方面免除了对包中协议内容的解析,其速度明显地由于需要解析信息的方法。同时,该方法可以解构成三个模块(数据获取、数据预处理以及数据分析)并行实现,因此具备了实时性的优势。

该方法实现了对较长攻击数据流的99\%检测率。不足之处是,对于较短的攻击数据流会漏检,这是来自于该方法对数据进行批处理。

\textbf{Automated Attack Discovery in TCP Congestion}

Samuel Jero等人(2018)提出了一种基于有限状态机的TCP拥塞攻击发现方法。该方法受基于模型的测试(MBT)和模糊测试启发,利用TCP拥塞控制中的有限状态机作为模型指导模糊测试,从而发现新的攻击方式。

首先,该方法在有限状态机中寻找所有的可能攻击路线;然后将该路线转换成实际的攻击动作与数据包。这两部分别被称作抽象攻击策略与具体攻击策略。接着,作者在虚拟网络中进行了攻击测试,以判断该具体攻击策略是否确实显著地影响了网络流量。通过对可行的攻击策略进行分类,作者总结了11种TCP拥塞攻击,其中8种是目前文献中未被报道过的。

该方法的缺点在于,目前对于攻击的分析仍需通过日志文件由人工进行,即便可以在一定程度上自动化分析一部分,约11\%的攻击方案仍需要人工分析。

Acknowledgement-based attacks - 通过恶意修改/伪造TCP ACK包使得系统陷入拥塞

Fuzzing技术是一种基于黑盒(或灰盒)的测试技术,通过自动化生成并执行大量的随机测试用例来发现产品或协议的未知漏洞。随着计算机的发展,Fuzzing技术也在不断发展。
\href{https://zhuanlan.zhihu.com/p/43432370}{Fuzzing技术总结(Brief Surveys on Fuzz Testing)}

modelbased testing (MBT) - an approach that generates effective test cases based on a model of the program.

We propose to automatically find manipulation attacks by guiding a protocol fuzzer with concrete attack actions derived from abstract attack strategies, which are obtained using a model-guided technique inspired by modelbased testing.

\textbf{Entropy-score}

Akshat Gaurav等人(2017)提出了一种基于熵和评分机制的层次性DDoS攻击检测方法。该方法基于这样一个观察:在DDoS攻击期间,服务器接收到的包中源IP地址的熵值会比平常要高。该方法同时解决了如何将DDoS攻击与特定时段用户的大量访问进行区分的问题。

接收到的数据包会按照预先设置好的一些中心值进行分组,分组的依据是到源IP该值的距离。熵由各组的经验概率计算;分数由当前该组的概率与历史记录中的概率相除得到。如果最近一段时间内源IP的熵大于阈值,那说明系统目前可能遭遇DDoS攻击,但也有可能是用户的正常访问;如果小于阈值,当前数据会被记录下来更新分数表。接着,计算各组的分数,如果分数高于阈值,说明发生了DDoS攻击,不然则是正常的用户访问。

该方法综合了利用熵和计分方法的优势,做到了较低的虚警率。遗憾的是作者没有提及检测的精确率,也没有提及对于DDoS攻击和大量用户访问之间区分性有多好。但这种分析的方法在一众机器学习方法之中令人耳目一新。

\textbf{Traceroute-based}

K. Sakuma等人(2017)介绍了一种应对新型DDoS攻击——目标链路泛洪攻击(Target link flooding attack)的方法。该方法利用traceroute进行检测。目标链路泛洪攻击不直接针对目标区域发起DDoS攻击,而是对互联网中特定的链路进行攻击,从而将目标区域与其他区域隔离。

该方法的提出考虑到为实现这种攻击,攻击者需要对目标区域附近的网络拓扑结构进行探测。这就需要攻击者发送大量traceroute包,且这些包会集中在距离目标服务器若干跳的范围内。因此通过分析traceroute及其距离目标服务器的跳数即可对该类型DDoS攻击进行探测。

首先,距离目标服务器一定范围内会部署一定量的监控器,对traceroute进行记录,并将包按照距离目标服务器的跳数进行分类;总分析服务器收集各个分散的监控器的信息,计算分数。如果两个相邻时刻的分数相差超过阈值,那么就会发出攻击警报。

这种按跳数进行恩类的思想是这篇文章的亮点所在。它克服了只利用traceroute数量变化的方法的缺点,更加健壮。但是需要部署额外的探测器是一个问题,作者没有在文章中介绍探测器数目以及部署的策略对该方法的影响。

\bibitem{fieldbus} \href{https://www.felser.ch/download/iec_61158_fieldbus.html}{IEC 61158 Fieldbus}

\textbf{FUTURE WORK}

3. APP layer too many various kinds - middleware; no many methods on direct DL layer and PHY;
    multiple layer co-operate

1. 应用层IDS缺少统一性方案

在本文涉及的文章中,面对DNS服务、HTTPS协议、浏览器插件广告插入以及电子邮件等,解决的方案都是特定的、针对性的。这是受应用层特性决定的:由于应用层希望为用户提供多种多样的服务,因此应用的数据格式高度分化,很难找到统一的监测方案。是否能找到一种通用性的解决方案为应用层提供更好的安全保障呢?

受到HTTPS下层的TLS协议启发,由于上层注定无法统一,因此从其下层进行处理是更好的策略。目前有许多流行的中间件,比如Google的QUIC协议,利用这些中间件为上层应用层服务提供更好的安全性是很有意义的研究方向。

是否可以设计一种能满足多种不同应用层需求的协议?它可以灵活的组装各种模块,就像IPSec中的IKE一样,可以为差异很大的不同应用提供服务;而它又可以有效地被IDS检测和分析。这种\textbf{中间件安全}引人思考。

2. 针对数据链路层以及物理层的监测方案较少

在寻找合适的文章过程中,一个现象是很少有针对数据链路层或物理层进行安全监测的方案。这样的检测是有益的,考虑到

\begin{enumerate}
    \item \textbf{应对信息受限情况}

    正如Peter Schneider等人这篇文章所述,在安全检测时,有时对上层协议不能获得完全的了解和知识,一个通用性的解决方案可以处理这样的问题;同时通用性也避免了重复开发的花费。

    \item \textbf{处理上层加密情况}

    TLS被越来越多的应用所使用;IPSec也可以对数据包进行保护;但是他们有共同的特点:数据内容被加密。这虽然为用户信息安全提供了一定的保障;但是也为攻击者提供了方便——加密的数据意味着许多方法不能被利用。而直接在数据链路层或物理层展开监测可以处理加密信息的安全监测。

    \item \textbf{对下层攻击的预防}

    在安全领域中,很常见的一种攻击方法是下层攻击。虽然本层的防御体系已经十分完善,但是对于来自下层的攻击还是力有不逮。如果,攻击者直接对电磁信号进行调整或伪造,或者在链路层中对帧进行修改,都可能会使上层的防御手段失效。
\end{enumerate}

但由于数据链路层以及物理层所包含的语义信息较少,各种检测方法有时难以实现较高的准确率。但这不代表这一层上不能够或不应该进行入侵检测。

3. 各层之间的协作

受到DPI启发,充分利用各个层之间的信息也对实现更好的入侵检测是很有帮助的。Peter Schneider等人(autoencoder)文章中所述,一些针对工业联网设备的攻击是控制逻辑上的攻击,也即其内容本身可能符合安全标准,但是多个包连续的组合会产生特定的攻击。这样的攻击,如果能通过最初对字节流的检测,然后更进一步对报文内容进行分析处理,相比可以更进一步提高准确性,并降低虚警率。

如果凭借本层信息不能完全确定,那么可以将本层的分析结果传送到上层的检测系统中,用于辅助进一步的分析。这样的协作模式可以对检测系统的提升有很大帮助。但是这同样会面里一个问题:开销的增长。面对越来越庞大的数据流,多层协作的IDS能否实时地对系统进行保护?这需要平衡协作的程度与开销,达到一个令人满意的程度。
